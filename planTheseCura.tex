%% ---------------------------------------------------------------------
%% Copyright 2015, Thales, IGN, Rémi Cura
%% 
%% This file describes the plan for the thesis manuscript
%% ---------------------------------------------------------------------
 
\documentclass{article}
\begin{document}
 
 
 % short plan
 \begin{enumerate}
 \item Introduction : what context , problem, goal , stucture of manuscript
 	 \begin{enumerate}
 	 	\item context : more city, bigger, more difficult to manage
 	 	\item problem : need tools to manage city
 	 	\item goal : automatic tools to create/update city model
 	 	\item structure of thesis : 
 	 		tools will be based on reality, aka sensing, which need management.
 	 		We need a way to model city.
			We need a way to ensure modeling is coherent with sensing
 	 \end{enumerate}
 	 
 \item Chapter 1 : Dealing with sensing : Point Cloud Server
 	 \begin{enumerate} 
  	 	\item what is sensing : introduction to point clouds
  	 	\item Introduction : context : more sensing, problem : volume, messy , goal : easy management, higher level information extraction. structure : state of the art, concept, loading, storing, filtering, outputing, processing
  	 	\item Short State of the art (small extension of paper)
  	 	\item Concept of Using RDBMS to manage point cloud
  	 	\item Basic operations
  	 	\begin{enumerate} 
  	 		\item loading
		  	 \item storing
		  	 \item filtering
		  	 \item exporting	
  	 	\end{enumerate}
  	 	\item processing
  	 		\begin{enumerate} 
  	 	  	 	\item LOD
  	 			\item extracting information	
  	 	  	\end{enumerate}
  	 	\item conclusion
  	 		The system manages huge sensing, and extract higher level oberservations suitable for city modeling. However, information too noisy and un-reliable to adopt reconstruction strategy. Better perform inverse procedural modeling --> Need procedural modeling.
  	 \end{enumerate}
  	 
 \item Chapter 2 : Generating Street : StreetGen
 	\begin{enumerate}
 	\item Introduction (procedural generation, tradeoff between model expressibility/complexity? Problem : diversity of streets. Solution : good first guess, editable, + natively easily extensible data model)
 	\item State of the art (only focused on road modelling, short (about 20 papers))
 	\item modelling streets (explaining the arc/line model , and how to compute it)
 	\item modelling a city : explains the choice of RDBMS use (scaling), and the general structure of the datamodel (topo network + results)
 	\item interactive changes (why, how, proof of usefulness)
 	\end{enumerate}
 	
 \item Chapter 3 : adapting model to sensing : Optimisation
 	\begin{enumerate}
 	\item Introduction : context : if successful, extraordinary powerful, but increadibely difficult. Boils down to Inverse procedural modeling, a pervasive problem, but a newly formulated one. Solution : wide state of the art, solution with known methods, perspective with un-tested methods
 	\item State of the Art : massive
 	\item Solving a simplified problem with a tried and tested solution : optimisation of a non linear least square problem
 	\item Proposition to solve the full problem (hypergraph matching)
 	\item conclusion : basic optimisation works, but won't suffice for complex version of the problem (aka with more than a few parameters). Still works to do for Inverse Procedural Modeling
 	\end{enumerate}
 	
 \item Conclusion : what ought to be improved
 \item Perspective
 \end{enumerate}

\end{document}



